\section{Modelos}
Dados los valores $R$, cantidad de filas del tablero y $C$, cantidad de columnas del tablero, \'este se model\'o como un arreglo de $R * C$ posiciones. Por comodidad, nos referiremos a cada posici\'on del tablero como si se tratase de una matriz.

Los estados del juego se modelaron como los distintos tableros posibles, y las transiciones son las columnas en donde es posible colocar la pr\'oxima ficha. Por ejemplo, el tablero vacío es el estado inicial. Desde este, hay tantas acciones posibles como columnas donde se puedan ubicar fichas. Cada una de estas acciones nos lleva a un estado nuevo, que es el tablero resultante de colocar una ficha en la columna correspondiente a la acción.

Luego, se definieron tres tipos de jugador: Humano, Random y Q-Learner. Este \'ultimo ser\'a el objeto de estudio, midiendo su aprendizaje frente al jugador Random. 
Posteriormente, se brinda la posibiliad de jugar contra el jugador entrenado.
