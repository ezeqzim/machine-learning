\section{Modelos}

El tablero fue modelado como un arreglo de $row \times col$ posiciones. Pero nos referimos a cada posici\'on como \'indices de una matriz.

Los estados del juego son los tableros, y las transiciones son las columnas en donde es posible colocar la pr\'oxima ficha. Por ejemplo, el tablero vacío es el estado inicial. Desde este, hay tantas acciones posibles como columnas donde se puedan ubicar fichas. Cada una de estas acciones nos lleva a un estado nuevo, que es el tablero resultante de colocar una ficha en la columna correspondiente a la acción.

Definimos tres tipos de jugador: Humano, Random y Q-Learning. Este \'ultimo ser\'a objeto de estudio, midiendo su aprendizaje frente a Random. Luego se da la posibiliad de jugar contra el jugador entrenado.
