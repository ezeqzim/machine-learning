\documentclass[10pt,a4paper]{article}
\usepackage[left=3cm, right=3cm]{geometry}
\usepackage[utf8]{inputenc} % para poder usar tildes en archivos UTF-8
\usepackage[spanish]{babel} % para que comandos como \today den el resultado en castellano
%\usepackage{a4wide} % márgenes un poco más anchos que lo usual
\usepackage{caratula}
\usepackage{amsmath}
\usepackage{colortbl}
\usepackage{tikz}
\usepackage{slashbox}
\usepackage{hhline}
%\usepackage{algpseudocode}
%\usepackage{algorithm}
\usepackage{enumerate}
%\usepackage{dsfont}
\usepackage{tabularx}
\usepackage[nomove]{cite}
%\usepackage[nottoc]{tocbibind}
\usepackage{hyperref}
\hypersetup{
    linktoc=all,     %set to all if you want both sections and subsections linked
    linkcolor=blue,  %choose shttps://www.sharelatex.com/project/542da47afb0e80fe43643f61ome color if you want links to stand out
}
\input{page.layout}
\newcommand{\ig}[2]{\begin{figure}\centering\includegraphics[width=15cm]{#1}\caption{#2}\end{figure}}

\begin{document}
%\newcolumntype{L}[1]{<{\raggedleft\arraybackslash}p{#1}}
\renewcommand{\labelitemii}{$\--$}

\titulo{Trabajo Práctico 2}
\subtitulo{\emph{Aprendizaje por Refuerzos}}

\fecha{\today}

\materia{Aprendizaje Automático}

\integrante{Donatucci, Nicolás Andres}{263/13}{nadonatucci@gmail.com}
\integrante{Noriega, Francisco José}{660/12}{frannoriega.92@gmail.com}
\integrante{Zimenspitz, Ezequiel}{155/13}{ezeqzim@gmail.com}
% Pongan cuantos integrantes quieran
%\keywords{Cálculo de autovectores, Método de la potencia, PageRank, HITS, In-Deg}
\maketitle
\newpage
\thispagestyle{empty}
\vfill
\vspace{3cm}
\tableofcontents
\newpage
\setcounter{page}{1}

\section{Introducción}
Este trabajo tiene como alcance la implementaci\'on y experimentaci\'on de la t\'ecnica de  \emph{Q-learning} de aprendizaje por refuerzos, que consisti\'o en generar un sistema que aprenda a jugar al conocido juego \emph{4 en l\'inea}.
Esto implic\'o la necesidad de realizar un modelo del juego, con sus correspondientes estados y transiciones.
Luego se procedi\'o a realizar experimentos, que consistieron en comparar los resultados de los \emph{Q-learners} en distintos escenarios de aprendizaje, con el objetivo de comprender cómo se realiza el mismo.

\section{Modelos}

Una vez escogidos los atributos, se discutió qué clasificadores se probarían y qué combinaciones de parámetros para cada uno. Para realizar las corridas se utilizó el \texttt{gridSearch} de \texttt{sklearn}.

Se decidió utilizar los siguientes clasificadores, variando para cada uno los siguientes parámetros.

\textbf{Gaussian Naive Bayes Classifier}\\
\textbf{Multinomial Naive Bayes Classifier}
\begin{itemize}
	\item \texttt{alpha}: Additive (Laplace/Lidstone) smoothing parameter.\\
	\texttt{0.0, 0.5, 1.0}
	\item \texttt{fit\_prior}: Whether to learn class prior probabilities or not. If false, a uniform prior will be used.\\
	\texttt{true, false}
\end{itemize}
\textbf{Bernoulli Naive Bayes Classifier}
\begin{itemize}
	\item \texttt{alpha}: Additive (Laplace/Lidstone) smoothing parameter.\\
	\texttt{0.0, 0.5, 1.0}
	\item \texttt{fit\_prior}: Whether to learn class prior probabilities or not. If false, a uniform prior will be used.\\
	\texttt{true, false}
	\item \texttt{binarize}: Threshold for binarizing (mapping to booleans) of sample features. If None, input is presumed to already consist of binary vectors.\\
	\texttt{0.0}
\end{itemize}
\textbf{Random Forest Classifier}
\begin{itemize}
	\item \texttt{n\_estimators}: The number of trees in the forest.\\
	\texttt{5, 10, 15}
	\item \texttt{criterion}: The function to measure the quality of a split.\\
	\texttt{gini, entropy}
	\item \texttt{max\_features}: The number of features to consider when looking for the best split. If None, use all features.\\
	\texttt{None, sqrt, log2}
	\item \texttt{max\_depth}: The maximum depth of the tree. If None, then nodes are expanded until all leaves are pure or until all leaves contain less than \texttt{min\_samples\_split} samples.\\
	\texttt{None, 10, 25, 50}
	\item \texttt{bootstrap}: Whether bootstrap samples are used when building trees.\\
	\texttt{true, false}
\end{itemize}
%%TODO FRAN, 
%%Acá básicamente copy+paste del gridsearch y los parámetros que buscaste vos.
%%Basicamente armate un itemize o enumerate de los distintos clasificadores y para cada uno habla de los atributos que variamos explicando brevemente qué hacen.

Antes de comenzar se separó un 10\% de los mails (4500 mails $hm$ y 4500 mails $spam$) para usar como set de validación.

%%TODO EZE
%%Acá básicamente lo único que te pediría es que cuentes levemente como funciona gridsearch, más que nada para dejar claro que hace cross-validation, que no lo mencionamos nunca por ahora, especificando que la cross-validation no toca los datos de validación.

Luego se procedió a quedarse con la mejor combinación de parámetros para cada clasificador según \texttt{gridSearch}. Para esto se decidió utilizar el score $f1$. Luego se corrió cada uno de los clasificadores con su mejor combinación de parámetros sobre el set de validación.

%%TODO EZE
%%No se si hablar acá de las cosas que fallaron y hablar en general de las cosas implementativas o si hablar en la sección de resultados. Pero habría que mencionar lo que falló, por qué falló y hablar un poco de tiempos, básicamente hablar bien de arboles y compañía.
\definecolor{intvier}{RGB}{144, 224, 87}
\section{Experimentación y Resultados}
Antes de comenzar, se realizaron pruebas sobre el valor inicial de una nueva entrada en el diccionario $\{estado, accion\} \rightarrow \text{\emph{Q-valor}}$. Esto es, el valor de $Q$ para un estado no visitado.

Se comienza con un valor de 1.0. A medida van pasando las distintas iteraciones, se observa que el Q-learner aprende contra un Random, es decir, comienza a ganar un porcentaje mayor de veces.

Colocando un valor random entre 0 y 1, el entrenamiento es errático. No se observaba un crecimiento mantenido en el porcentaje de partidas ganadas respecto del Random.

Usando 0.0, funciona igual que con 1.0, pero con un porcentaje de victorias del Q-Learner considerablemente mayor.
Por esta raz\'on, se decidi\'o utilizar 0.0 como valor inicial de $Q$ para cada estado y acci\'on correspondiente.\\

Luego, se llev\'o a cabo una serie de ejecuciones, variando los tres hiperpar\'ametros de la t\'ecnica de Q-Learning, para determinar c\'omo afectan al aprendizaje. 
En estas ejecuciones, se entrenaba un Q-Learner contra otro Q-Learner, y se somet\'ia al mejor de los dos, a jugar contra un jugador Random, midiendo la cantidad de victorias contra este \'ultimo. Los resultados fueron los siguientes:

\begin{center}
\begin{tabular}{|c||c|c|c|c|}
	\hline
	\backslashbox{$\alpha$}{$\gamma$} & 0.6 & 0.7 & 0.8 & 0.9\\
	\hline
	\hline
	0.2 & 65.63\% & 65.62\% & 65.72\% & \cellcolor{intvier}66.17\% \\
	\hline
	0.3 & 65.59\% & 65.70\% & 65.58\% & 66.14\% \\
	\hline
	0.4 & 65.52\% & 65.71\% & 65.62\% & 66.01\% \\
	\hline
	0.5 & 65.83\% & 65.40\% & 65.74\% & 65.70\% \\
	\hline
\end{tabular}\\
Porcentaje de victorias con $\varepsilon=0.1$
\end{center}
\begin{center}
\begin{tabular}{|c||c|c|c|c|}
	\hline
	\backslashbox{$\alpha$}{$\gamma$} & 0.6 & 0.7 & 0.8 & 0.9\\
	\hline
	\hline
	0.2 & 59.60\% & 59.91\% & 59.95\% & 60.59\% \\
	\hline
	0.3 & 59.67\% & 59.75\% & 59.82\% & \cellcolor{intvier}60.78\% \\
	\hline
	0.4 & 60.09\% & 59.85\% & 59.74\% & 60.75\% \\
	\hline
	0.5 & 59.63\% & 59.90\% & 59.80\% & 60.74\% \\
	\hline
\end{tabular}\\
Porcentaje de victorias con $\varepsilon=0.2$
\end{center}
\begin{center}
\begin{tabular}{|c||c|c|c|c|}
	\hline
	\backslashbox{$\alpha$}{$\gamma$} & 0.6 & 0.7 & 0.8 & 0.9\\
	\hline
	\hline
	0.2 & 55.27\% & 55.07\% & 55.37\% & 56.60\% \\
	\hline
	0.3 & 55.32\% & 55.40\% & 55.39\% & 56.55\% \\
	\hline
	0.4 & 54.83\% & 55.10\% & 55.78\% & 56.89\% \\
	\hline
	0.5 & 54.86\% & 54.97\% & 55.48\% & \cellcolor{intvier}56.90\% \\
	\hline
\end{tabular}\\
Porcentaje de victorias con $\varepsilon=0.3$
\end{center}
\begin{center}
\begin{tabular}{|c||c|c|c|c|}
	\hline
	\backslashbox{$\alpha$}{$\gamma$} & 0.6 & 0.7 & 0.8 & 0.9\\
	\hline
	\hline
	0.2 & 51.62\% & 51.94\% & 52.05\% & 53.74\% \\
	\hline
	0.3 & 51.73\% & 51.79\% & 52.23\% & 55.51\% \\
	\hline
	0.4 & 51.84\% & 51.42\% & 52.38\% & \cellcolor{intvier}53.76\% \\
	\hline
	0.5 & 51.88\% & 51.74\% & 52.40\% & 53.73\% \\
	\hline
\end{tabular}\\
Porcentaje de victorias con $\varepsilon=0.4$
\end{center}

Como se puede observar, a mayor valor de $\varepsilon$, peores los resultados obtenidos. Por lo tanto, se realizaron una nueva serie de ejecuciones que obtuvieron los siguientes resultados:
\begin{center}
\begin{tabular}{|c||c|c|c|}
	\hline
	\backslashbox{$\alpha$}{$\gamma$} & 0.0 & 0.1 & 1.0\\
	\hline
	\hline
	0.0 & - & - & 49.76\%\\
	\hline
	0.1 & - & - & \cellcolor{intvier}77.20\%\\
	\hline
	1.0 & 74.43\% & 74.34\% & 55.73\%\\
	\hline
\end{tabular}\\
Porcentaje de victorias con $\varepsilon=0$
\end{center}

En vista de estos resultados, se decidi\'o utilizar $\varepsilon = $ 0.0, $\alpha = $ 0.1 y $\gamma = $ 1.0 para los dem\'as experimentos.

Para todos los experimentos presentados a continuación, se entrenó a un Q-Learner contra diversos adversarios, a lo largo de un millón de iteraciones.

\subsection{Aprende el Q-Learner contra un Random?}

En un principio queremos saber si el Q-Learner aprende. Es decir, si tras un número considerable de partidas jugadas contra un Random, consigue aumentar su porcentaje de partidas ganadas. Para probar ésto entrenamos a un Q-Learner contra un Random durante un millón de iteraciones, graficando los resultados \footnote{Para graficar los resultados de los experimentos, se presentan dos tipos de gráficos. Primero, se cuenta la cantidad de partidas ganadas por cada jugador en ventanas de 500 partidas y se calcula el porcentaje que cada uno ganó. Luego, se presenta una cuenta similar pero siempre sobre el total de partidas, en lugar de sobre ventanas de 500.}.

Nuestra conjetura es que el Q-Learner va a aprender. Es decir, a medida vayan pasando las iteraciones, el porcentaje de partidas ganadas va a aumentar. También esperamos que tras aproximadamente 200000 iteraciones ya comience a notarse la mejoría.

Se presentan a continuación los resultados del experimento. % DECIR CON QUE PARAMETROS SE CORRIO

\ig{Imagenes/Results/exp1/zero_any_move/zero_any_move.png}

\ig{Imagenes/Results/exp1/zero_any_move/zero_any_move_acum.png}

Se puede ver que con el correr del entrenamiento, el Q-Learner gana cada vez m\'as partidas de las 500. A su vez, el porcentaje acumulado aumenta consistentemente.

Queda claro ver que al principio el Random gana algunos partidos m\'as que el Q-Learning, pero con muy pocas iteraciones, este comienza a ganar mayor cantidad de partidas.

La curva de aprendizaje es creciente. Podemos observar que tras 200000 iteraciones el Q-Learner está ganando alrededor del 65\% de las partidas. Luego, tras las otras 800000 iteraciones, el porcentaje está alrededor del 75\%. Nos interesa ver por qué el ritmo decrece.

Para observar el comportamiento del Q-Learner, jugamos partidas contra él para ver qué estrategia utiliza. Su estrategia es apilar fichas en la columna de más a la derecha. Si uno intenta bloquearlo, sigue poniendo fichas en esa columna hasta que se llene. Tras llenar esa columna, ya no es tan fijo lo que hace. Debido a que está jugando contra un Random, esto le debe traer buenos resultados un número interesante de veces.

Nuestra conjetura es que la gran mayoría de las veces que gana, gana con esa estrategia, por lo tanto dándole un gran valor de Q a esos estados y transiciones. Además, el Random no es un oponente muy interesante y conjeturamos que no le da la oportunidad de probar nuevos estados interesantes.

Pero viendo esto, falta aún ver por qué pierde. La respuesta a la que llegamos es que en parte se debe al coeficiente de exploración. Es decir, si explora quizás pierde. Una vez se desvía de su estrategia ganadora, vemos que su desempeño empeora. Nuestra conjetura es que, probablemente la mayor parte de sus victorias son con la estrategia de la columna derecha. Entonces no está aprendiendo mucho. Simplemente gana muchas partidas con una estrategia poco inteligente. Entonces le lleva mucho tiempo aprender cosas nuevas, porque está jugando contra un jugador muy poco inteligente.

Con esto en mente, decidimos ver qué ocurre si lo entrenamos contra un oponente más inteligente.


\subsection{Experimento 2}


\subsection{Experimento 3}

Este experimento consiste en entrenar un Q-Learning frente a otro. Lo ejecutamos durante un mill\'on de iteraciones (hasta aqu\'i igual que el experimento 2). Luego, un nuevo Q-Learning entrena frente al que obtuvo el mayor win\_percentage\footnote{Fue una decisi\'on completamente arbitratria} de los anteriores, ejecutamos durante un mill\'on de iteraciones y graficamos los resultados. Finalmente, este \'ultimo compite contra un Random durante la misma cantidad de iteraciones y graficamos los resultados.

Dada la pregunta planteada para este experimento, esperamos que la experiencia que el tercer Q-Learning obtenga al entrenar contra un jugador experimentado, lo ayude a vencer en mayor proporcici\'on a un Random en forma consistente y que este a un nivel similar al de un Humano.


\newpage
\section{Discusión}


\end{document}
