\section{Introducción}
Este trabajo tiene como alcance la implementaci\'on y experimentaci\'on de la t\'ecnica de  \emph{Q-learning} de aprendizaje por refuerzos, que consisti\'o en generar un sistema que aprenda a jugar al conocido juego `\emph{4 en l\'inea}.
Esto implic\'o la necesidad de realizar un modelo del juego, con sus correspondientes estados y transiciones. 
Luego se procedi\'o a realizar experimentos, que consistieron en comparar los resultados de los \emph{Q-learners} en distintos escenarios de aprendizaje, con el objetivo de comprender cómo se realiza el mismo.
