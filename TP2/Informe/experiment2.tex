\subsection{Aprende un Q-Learner entrenado contra otro Q-Learner?}

Queremos ver si, al entrenar un Q-Learner contra un oponente un poco más inteligente, se obtienen mejores resultados. Este experimento consiste en entrenar un Q-Learner contra otro. Lo ejecutamos durante un mill\'on de iteraciones y graficamos los resultados. Luego, cada uno de estos compite contra un Random durante la misma cantidad de iteraciones y graficamos los resultados.

Los dos Q-Learners comienzan siendo muy poco inteligentes. A medida vayan pasando las iteraciones, ambos deberían ir aprendiendo de forma conjunta. Es decir, se deberían ir haciendo mejores a un ritmo parejo. De esta forma conjeturamos que van a comenzar a explorar estados más interesantes, con partidas un poco más largas. En el mejor caso, se podría llegar a ver que comiencen a empatar, si ambos aprendieron a jugar muy bien.

Por eso conjeturamos que, una vez entrenado el Q-Learner, va a obtener mejores resultados contra un Random que los obtenidos en el experimento anterior, cuando entrenó únicamente contra un Random.

\textcolor{red}{CHAMUYO}
