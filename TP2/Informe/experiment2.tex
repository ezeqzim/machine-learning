\subsection{¿Aprende un Q-Learner entrenado contra otro Q-Learner?}

El enfoque de este experimento fue observar cómo se comportan dos Q-Learner al entrenar compitiendo entre sí, para determinar si los resultados obtenidos indicaban una mejoría en sus estrategias de juego.
Luego, cada uno de éstos compitió contra un jugador Random durante la misma cantidad de iteraciones y se graficaron los resultados.

\ig{Imagenes/Results/exp2/normal/normal_training.png}{Porcentaje de victorias de cada Q-Learner en ventanas de 500 iteraciones}

\ig{Imagenes/Results/exp2/normal/normal_training_acum.png}{Porcentaje de victorias de cada Q-Learner a lo largo de las iteraciones}

Observando el gráfico de los resultados tomados en ventanas de 500 iteraciones, se puede ver que las victorias de ambos fluctuan alrededor del 50\%, alternando quién es el que efectivamente posee el mayor porcentaje.

Entonces se procedió a observar si el Q-Learner entrenado contra otro Q-Learner tenía un mejor desempeño contra un jugador Random. Para esto se utilizó un coeficiente de exploración igual a 0. Se obtuvieron los siguientes resultados.

\ig{Imagenes/Results/exp2/normal/normal_p1_test.png}{Porcentaje de victorias de cada Q-Learner en ventanas de 500 iteraciones}

\ig{Imagenes/Results/exp2/normal/normal_p1_test_acum.png}{Porcentaje de victorias de cada Q-Learner a lo largo de las iteraciones}

Se puede observar que el Q-Learner gana por encima del 90\% de las veces. Los resultados no son mucho mejores que los del experimento anterior, por lo que se intentó jugar contra el Q-Learner. Esta vez su estrategia no era siempre ubicar las fichas en la columna de más a la derecha. De todas formas, seguía sin ganar cuando podía ganar y sin bloquear cuando podía perder. Pero su estrategia era un poco más variada.

Nuestra conjetura es que se obtuvieron mejores resultados contra los humanos porque el enfrentamiento entre dos Q-Learners representa un desafío mayor que enfrentar a un jugador Random, dado que al jugador Random se le puede vencer con facilidad, en la mayoría de los casos, simplemente ubicando fichas en una misma columna.
