\definecolor{intvier}{RGB}{144, 224, 87}
\section{Experimentación y Resultados}
Antes de comenzar, se realizaron pruebas sobre el valor inicial de una nueva entrada en el diccionario $\{estado, accion\} \rightarrow \text{\emph{Q-valor}}$. Esto es, el valor de $Q$ para un estado no visitado.

Se comienza con un valor de 1.0. A medida van pasando las distintas iteraciones, se observa que el Q-learner aprende contra un Random, es decir, comienza a ganar un porcentaje mayor de veces.

Colocando un valor random entre 0 y 1, el entrenamiento es errático. No se observaba un crecimiento mantenido en el porcentaje de partidas ganadas respecto del Random.

Usando 0.0, funciona igual que con 1.0, pero con un porcentaje de victorias del Q-Learner considerablemente mayor.
Por esta raz\'on, se decidi\'o utilizar 0.0 como valor inicial de $Q$ para cada estado y acci\'on correspondiente.\\

Luego, se llev\'o a cabo una serie de ejecuciones, variando los tres hiperpar\'ametros de la t\'ecnica de Q-Learning, para determinar c\'omo afectan al aprendizaje. 
En estas ejecuciones, se entrenaba un Q-Learner contra otro Q-Learner, y se somet\'ia al mejor de los dos, a jugar contra un jugador Random, midiendo la cantidad de victorias contra este \'ultimo. Los resultados fueron los siguientes:

\begin{center}
\begin{tabular}{|c||c|c|c|c|}
	\hline
	\backslashbox{$\alpha$}{$\gamma$}& 0.6 & 0.7 & 0.8 & 0.9\\
	\hline
	\hline
	0.2 & 65.63\% & 65.62\% & 65.72\% & \cellcolor{intvier}66.17\% \\
	\hline
	0.3 & 65.59\% & 65.70\% & 65.58\% & 66.14\% \\
	\hline
	0.4 & 65.52\% & 65.71\% & 65.62\% & 66.01\% \\
	\hline
	0.5 & 65.83\% & 65.40\% & 65.74\% & 65.70\% \\
	\hline
\end{tabular}\\
Porcentaje de victorias con $\varepsilon=0.1$
\end{center}
\begin{center}
\begin{tabular}{|c||c|c|c|c|}
	\hline
	\backslashbox{$\alpha$}{$\gamma$}& 0.6 & 0.7 & 0.8 & 0.9\\
	\hline
	\hline
	0.2 & 59.60\% & 59.91\% & 59.95\% & 60.59\% \\
	\hline
	0.3 & 59.67\% & 59.75\% & 59.82\% & \cellcolor{intvier}60.78\% \\
	\hline
	0.4 & 60.09\% & 59.85\% & 59.74\% & 60.75\% \\
	\hline
	0.5 & 59.63\% & 59.90\% & 59.80\% & 60.74\% \\
	\hline
\end{tabular}\\
Porcentaje de victorias con $\varepsilon=0.2$
\end{center}
\begin{center}
\begin{tabular}{|c||c|c|c|c|}
	\hline
	\backslashbox{$\alpha$}{$\gamma$}& 0.6 & 0.7 & 0.8 & 0.9\\
	\hline
	\hline
	0.2 & 55.27\% & 55.07\% & 55.37\% & 56.60\% \\
	\hline
	0.3 & 55.32\% & 55.40\% & 55.39\% & 56.55\% \\
	\hline
	0.4 & 54.83\% & 55.10\% & 55.78\% & 56.89\% \\
	\hline
	0.5 & 54.86\% & 54.97\% & 55.48\% & \cellcolor{intvier}56.90\% \\
	\hline
\end{tabular}\\
Porcentaje de victorias con $\varepsilon=0.3$
\end{center}
\begin{center}
\begin{tabular}{|c||c|c|c|c|}
	\hline
	\backslashbox{$\alpha$}{$\gamma$}& 0.6 & 0.7 & 0.8 & 0.9\\
	\hline
	\hline
	0.2 & 51.62\% & 51.94\% & 52.05\% & 53.74\% \\
	\hline
	0.3 & 51.73\% & 51.79\% & 52.23\% & 55.51\% \\
	\hline
	0.4 & 51.84\% & 51.42\% & 52.38\% & \cellcolor{intvier}53.76\% \\
	\hline
	0.5 & 51.88\% & 51.74\% & 52.40\% & 53.73\% \\
	\hline
\end{tabular}\\
Porcentaje de victorias con $\varepsilon=0.4$
\end{center}

Como se puede observar, a mayor valor de $\varepsilon$, peores los resultados obtenidos. Por lo tanto, se realizaron una nueva serie de ejecuciones que obtuvieron los siguientes resultados:
\begin{center}
\begin{tabular}{|c||c|c|c|}
	\hline
	\backslashbox{$\alpha$}{$\gamma$}& 0.0 & 0.1 & 1.0\\
	\hline
	\hline
	0.0 & - & - & 49.76\%\\
	\hline
	0.1 & - & - & \cellcolor{intvier}77.20\%\\
	\hline
	1.0 & 74.43\% & 74.34\% & 55.73\%\\
	\hline
\end{tabular}\\
Porcentaje de victorias con $\varepsilon=0$
\end{center}

En vista de estos resultados, se decidi\'o utilizar $\varepsilon = $ 0.0, $\alpha = $ 0.1 y $\gamma = $ 1.0 para los dem\'as experimentos.

\subsection{Aprende el Q-Learner contra un Random?}

En un principio queremos saber si el Q-Learner aprende. Es decir, si tras un número considerable de partidas jugadas contra un Random, consigue aumentar su porcentaje de partidas ganadas. Para probar ésto entrenamos a un Q-Learner contra un Random durante un millón de iteraciones, graficando los resultados \footnote{Para graficar los resultados de los experimentos, se presentan dos tipos de gráficos. Primero, se cuenta la cantidad de partidas ganadas por cada jugador en ventanas de 500 partidas y se calcula el porcentaje que cada uno ganó. Luego, se presenta una cuenta similar pero siempre sobre el total de partidas, en lugar de sobre ventanas de 500.}.

Nuestra conjetura es que el Q-Learner va a aprender. Es decir, a medida vayan pasando las iteraciones, el porcentaje de partidas ganadas va a aumentar. También esperamos que tras aproximadamente 200000 iteraciones ya comience a notarse la mejoría.

Se presentan a continuación los resultados del experimento. % DECIR CON QUE PARAMETROS SE CORRIO

\ig{Imagenes/Results/exp1/zero_any_move/zero_any_move.png}

\ig{Imagenes/Results/exp1/zero_any_move/zero_any_move_acum.png}

Se puede ver que con el correr del entrenamiento, el Q-Learner gana cada vez m\'as partidas de las 500. A su vez, el porcentaje acumulado aumenta consistentemente.

Queda claro ver que al principio el Random gana algunos partidos m\'as que el Q-Learning, pero con muy pocas iteraciones, este comienza a ganar mayor cantidad de partidas.

La curva de aprendizaje es creciente. Podemos observar que tras 200000 iteraciones el Q-Learner está ganando alrededor del 65\% de las partidas. Luego, tras las otras 800000 iteraciones, el porcentaje está alrededor del 75\%. Nos interesa ver por qué el ritmo decrece.

Para observar el comportamiento del Q-Learner, jugamos partidas contra él para ver qué estrategia utiliza. Su estrategia es apilar fichas en la columna de más a la derecha. Si uno intenta bloquearlo, sigue poniendo fichas en esa columna hasta que se llene. Tras llenar esa columna, ya no es tan fijo lo que hace. Debido a que está jugando contra un Random, esto le debe traer buenos resultados un número interesante de veces.

Nuestra conjetura es que la gran mayoría de las veces que gana, gana con esa estrategia, por lo tanto dándole un gran valor de Q a esos estados y transiciones. Además, el Random no es un oponente muy interesante y conjeturamos que no le da la oportunidad de probar nuevos estados interesantes.

Pero viendo esto, falta aún ver por qué pierde. La respuesta a la que llegamos es que en parte se debe al coeficiente de exploración. Es decir, si explora quizás pierde. Una vez se desvía de su estrategia ganadora, vemos que su desempeño empeora. Nuestra conjetura es que, probablemente la mayor parte de sus victorias son con la estrategia de la columna derecha. Entonces no está aprendiendo mucho. Simplemente gana muchas partidas con una estrategia poco inteligente. Entonces le lleva mucho tiempo aprender cosas nuevas, porque está jugando contra un jugador muy poco inteligente.

Con esto en mente, decidimos ver qué ocurre si lo entrenamos contra un oponente más inteligente.


\subsection{¿Aprende un Q-Learner entrenado contra otro Q-Learner?}

El enfoque de este experimento fue observar cómo se comportan dos Q-Learner al entrenar compitiendo entre sí, para determinar si los resultados obtenidos indicaban una mejoría en sus estrategias de juego.
Luego, cada uno de éstos compitió contra un jugador Random durante la misma cantidad de iteraciones y se graficaron los resultados.

\ig{Imagenes/Results/exp2/gridsearch/gridsearch_training_acum.png}{Porcentaje de victorias de cada Q-Learner a lo largo de las iteraciones}

Ambos Q-Learners comienzan demostrando una baja inteligencia, ya que ambos realizan jugadas aleatorias al principio. Sin embargo, conforme avanzan las iteraciones, ambos comienzan a realizar mejores movimientos, y se puede observar que mejoran a un ritmo similar, dado que el porcentaje de victorias de cada uno tiende al 50\% al final, indicando que ambos tienen una estrategia ganadora.\\

Por los resultados obtenidos, se conjeturó que, una vez entrenado el Q-Learner, obtendrá más victorias contra el jugador Random que los obtenidos en el experimento anterior, cuando entrenó únicamente contra un jugador Random.
Dicha conjetura se condijo con los resultados obtenidos.

\ig{Imagenes/Results/exp2/gridsearch/gridsearch_p1_test_acum.png}{Porcentaje de victorias de cada Q-Learner a lo largo de las iteraciones}

\textcolor{red}{CHAMUYO}


\subsection{Aprende un Q-Learner inexperto contra un Q-Learner experimentado?}

\textcolor{red}{Al correr el experimento con los parametros $\varepsilon = $ 0.0, $\alpha = $ 0.1 y $\gamma = $ 1.0, los resultados fueron [INSERT RESULTADOS HERE].
Estos resultados demuestran que un Q-Learner sin entrenar puede vencer a un Q-Learner entrenado.
Se conjeturó que esto sucede porque no se favorece la exploración de nuevos estados (es decir, el Q-Learner solo explora cuando no conoce un estado, y realiza la misma jugada siempre si lo conoce), y porque el $\alpha$ es muy bajo, lo que indica que deja de aprender a un ritmo muy rápido. Lo que podría estar ocurriendo, es que el Q-Learner entrenado siempre intente realizar la misma estrategia, por lo que el nuevo Q-Learner sólo debe aprender a vencerla.
Una forma de validar que dicha conjetura es cierta, fue creando un nuevo jugador, el jugador Bobo, cuyas unico movimiento es poner una ficha en la última columna donde se pueda colocar; y verificando que éste tuviera resultados similares a los del Q-Learner cuando se enfrenta al jugador Random.
Los resultados son [INSERT RESULTADOS HERE].
Como se puede observar, el jugador Bobo juega igual de bien que el Q-Learner contra el jugador Random. Dicho resultado podría entonces explicar por que el Q-Learner experimentado pierde contra el Q-Learner sin entrenamieto: el experimentado consigue aprender una estrategia y siempre la aplica (y por eso, al igual que el jugador Bobo, siempre le gana al jugador Random), y el Q-Learner sin entrenamiento sólo debe aprender a vencerla.}

Tras el experimento anterior, nos surje la duda de qué pasaría si tomásemos uno de los Q-Learners entrenados del experimento anterior y lo pusieramos a jugar contra un Q-Learner novato. Aprenderá? En caso de que aprenda, qué pasaría si luego lo hiciésemos jugar contra un Random? Estará preparado para ganarle?

Este experimento consiste en entrenar un Q-Learning frente a otro. Lo ejecutamos durante un mill\'on de iteraciones (hasta aqu\'i igual que el experimento 2). Luego, un nuevo Q-Learning entrena frente al que obtuvo el mayor win\_percentage\footnote{Fue una decisi\'on completamente arbitratria} de los anteriores, ejecutamos durante un mill\'on de iteraciones y graficamos los resultados. Para esto, al Q-Learner ya entrenado le ponemos $epsilon$ en 0. De esta forma le eliminamos la exploración. Finalmente, este \'ultimo compite contra un Random durante la misma cantidad de iteraciones y graficamos los resultados.

A medida que vayan entrenando el experimentado contra el inexperto, queremos ver si el inexperto aprende algo. Es decir, si comienza a ganarle en algún momento al Q-Learner experimentado.

Cuando los Q-Learners comienzan, son casi un Random, dado que no tienen experiencia. Por eso, en el experimento anterior, ambos Q-Learners tuvieron en cierta medida experiencia de jugar contra Randoms, a pesar de haber entrenado entre sí. Para un Q-Learner novato que se enfrenta a uno ya entrenado, nuestra conjetura es que no recibirá la ‘‘experiencia'' de jugar contra un Random. Por eso, quizás al jugar luego contra un Random, no le vaya bien.

Presentamos a continuación los resultados del mejor Q-Learner experimentado (Rojo) contra el Q-Learner novato (Azul).

\ig{Imagenes/Results/exp3/gridsearch/gridsearch_best_p3_training.png}{COMPLETAR}

\ig{Imagenes/Results/exp3/gridsearch/gridsearch_best_p3_training_acum.png}{COMPLETAR}

\textcolor{red}{DONA, REVISAR LOS NUMEROS CON LOS NUEVOS GRAFICOS! RECORDAR QUE AHORA LOS PARAMS SON ALPHA = 0.1 EPSILON = 0.0 Y GAMMA = 1.0}

Podemos ver que durante las primeras 250000 iteraciones el Q-Learner experimentado siempre tuvo un porcentaje de partidas ganadas mayor, pero que el novato iba ganando cada vez más. Luego, comienzan a ganar más parejo. En el gráfico tomado de a 500 partidas, se pueden observar picos en dónde algúno mejora momentáneamente. En el acumulado se ve una curva más estable. El experimentado continúa aprendiendo, pero ya no explora más. De esto concluímos que jugando contra un oponente experimentado de todas formas aprende.

Vemos que en el gráfico de las partidas tomadas cada 500, alrededor de las 600000 iteraciones hay un período donde el porcentaje del novato crece y luego decrece. Conjeturamos que quizás descubrió una buena estrategia y al experimentado le llevó todas esas iteraciones ‘‘aprender cómo vencerla''. Creemos que esto se debe a que el experimentado sigue aprendiendo. Esto nos lleva a la pregunta de qué ocurriría si dejase de aprender.

Para esto repetimos el experimento, pero además de poner el $epsilon$ en 0 también ponemos el $alpha$ en 0. Observamos lo siguiente:

\ig{Imagenes/Results/exp3/no_learn/no_learn_best_p3_training.png}{COMPLETAR}

\ig{Imagenes/Results/exp3/no_learn/no_learn_best_p3_training_acum.png}{COMPLETAR}

Aquí podemos ver cómo el novato comienza a ganarle más veces al experimentado. Podemos ver que, al no aprender nada, el Experimentado comienza a perder cada vez más. Esto parece apoyar nuestra hipótesis de que en el experimento previo lo que sucedió fue que primero el novato aprendió una forma nueva de ganar y luego el experimentado aprendió a contrarrestarla.

