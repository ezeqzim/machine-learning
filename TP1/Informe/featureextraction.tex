\section{Extracción de atributos}

En un principio la idea original que se intentó seguir fue la de tomar las palabras más comunes de tanto los mails de $ham$ como los de $spam$. Para esto se utilizó el \texttt{CountVectorizer} de \texttt{sklearn}, utilizando el parámetro \texttt{max\_features}, que ordena las palabras del texto por frecuencia y sólo se queda con las más frecuentes (la cantidad pasada como parámetro). 

Para efectuar el split se utilizo una expresión regular que toma palabras con letras y números y las separa con los signos de puntuación. Se decidió ignorar las palabras que consistían únicamente de números. Al realizar una prueba se notó que las primeras en la lista de palabras más frecuentes eran palabras de \texttt{html} o palabras de los headers de los mails. También se notó que aparecía una palabra que consistía de una gran cantidad de letras "A" juntas, especialmente en $ham$. Tras investigar de qué mails provenían estas palabras, se descubrió que se debía a los archivos adjuntos.

Luego se intentó utilizar un parser de mails para quedarse únicamente con el body del mail. Durante esta etapa surgieron problemas con los mails $multipart$. Al pedirle el $payload$ éstos devolvían una lista de partes, de las cuales no se podía saber con precisión cual era el body. Tras una discusión entre los integrantes del grupo se decidió no utilizar esta herramienta. En primer lugar, dado que la palabra de las letras "A" era más común en $ham$, parecía algo interesante para mantener, entonces se decidió quedarse con los adjuntos. En cuanto a los headers, no se quitaron pero se decidió tomar todos los mails y tomar las 500 palabras más comunes en total. Esta decisión se basó en la idea de que de ésta forma se obtendrían las palabras más comunes más allá de los headers. Si bien quizás estos atributos como la frecuencia de la palabra $from$ podrían entorpecer un poco clasificadores como $knn$, los árboles de decisión podrían dejarlos de lado, por ejemplo, si se utiliza el criterio de la entropía, dado que no ayudarían a separar los datos. Más aún, serían los primeros en ser dejados de lado por métodos de reducción de dimensionalidad.

Entonces los atributos elegidos fueron las frecuencias de las 500 palabras más comunes en el total de los mails.